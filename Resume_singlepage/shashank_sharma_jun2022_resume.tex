	%-------------------------
% Resume in Latex
% Author : Shashank Sharma
%------------------------

\documentclass[letterpaper,10pt]{article}

\usepackage{latexsym}
\usepackage[empty]{fullpage}
\usepackage{titlesec}
\usepackage{marvosym}
\usepackage[usenames,dvipsnames]{color}
\usepackage{verbatim}
\usepackage{enumitem}
\usepackage[pdftex]{hyperref}
\usepackage{fancyhdr}
\usepackage{wasysym}
\usepackage{fontawesome}


\hypersetup{
  colorlinks   = true, %Colours links instead of ugly boxes
  urlcolor     = blue, %Colour for external hyperlinks
  linkcolor    = blue, %Colour of internal links
  citecolor   = blue %Colour of citations
}

\pagestyle{fancy}
\fancyhf{} % clear all header and footer fields
\fancyfoot{}
\renewcommand{\headrulewidth}{0pt}
\renewcommand{\footrulewidth}{0pt}

% Adjust margins
\addtolength{\oddsidemargin}{-0.5in}
\addtolength{\evensidemargin}{-0.5in}
\addtolength{\textwidth}{1in}
\addtolength{\topmargin}{-.5in}
\addtolength{\textheight}{1.0in}

\urlstyle{same}

\raggedbottom
\raggedright
\setlength{\tabcolsep}{0in}

% Sections formatting
\titleformat{\section}{
  \vspace{-4pt}\scshape\raggedright\large
}{}{0em}{}[\color{black}\titlerule \vspace{-5pt}]

%-------------------------
% Custom commands
% Main Headings like Company name with Tabular structure of Company & Place // designation & Duration
\newcommand{\resumeHeading}[4]{
  \vspace{-1pt}
    \begin{tabular*}{0.97\textwidth}{l@{\extracolsep{\fill}}r}
      \textbf{#1} & #2 \vspace{-2pt}\\ \vspace{1pt}
      \textit{\small#3} & \textit{\small #4} \\
    \end{tabular*}
      %\vspace{-5pt}
}
% Sub Headings e.g. Project Titles
\newcommand{\resumeSubheading}[1]{
      {\small\textbf{#1}} \\
      %\vspace{-5pt}
}

% For another role in same organization
\newcommand{\resumeSubheadingWithDate}[2]{
    \begin{tabular*}{0.97\textwidth}{l@{\extracolsep{\fill}}r}
      \textit{\small#1} & \textit{\small #2}\\
    \end{tabular*}
    \vspace{+2pt}
}

\newcommand{\resumeSubheadingNew}[1]{
      {\small{#1}} \\
      %\vspace{-5pt}
}

% Reducing Gap Between Sections
\newcommand{\resumeSection}[1]{
\vspace{-12pt}
\section{\textbf{#1}}
}

% Bullet list for details
\newcommand{\resumeItemListStart}{
\vspace{-6pt}
\begin{itemize}[leftmargin=14pt]
}
\newcommand{\resumeItemListEnd}{
\vspace{+6pt}
\end{itemize}
}

\newcommand{\resumeItem}[1]{
  \item\small{
      {#1 \vspace{-7pt}
      }
  }
}

\renewcommand{\labelitemii}{$\circ$}


%-------------------------------------------
%%%%%%  CV STARTS HERE  %%%%%%%%%%%%%%%%%%%%%%%%%%%%


\begin{document}
	
	
	
%----------HEADING-----------------
\begin{tabular*}{\textwidth}{l@{\extracolsep{\fill}}c@{\extracolsep{\fill}}r}
	\href{https://www.linkedin.com/in/ssharma1991/}{linkedin.com/in/ssharma1991} &\textbf{{\LARGE Shashank Sharma}} & \href{mailto:shashank.sharma.1991@outlook.com}{shashank.sharma.1991@outlook.com}\\
	\href{http://sharmashashank.com/}{sharmashashank.com} & \faHome\, Holland, Michigan \quad \phone\, 631-512-0029 & \href{https://github.com/ssharma1991}{github.com/ssharma1991}\\
\end{tabular*}
\vspace{+2pt}



%-----------EXPERIENCE-----------------
\resumeSection{Experience}
\resumeHeading
{Dematic, Kion Mobile Automation}{Holland, MI}
{Machine Learning Engineer (Perception, Localization, and Mapping)}{Sept 2020 -- Present}
\resumeItemListStart
\resumeItem{Developing autonomous vehicles for warehouse environments as part of the on-board software team.}
\resumeItem{Improved accuracy and robustness of the reflector extraction algorithm leading to sub-centimeter accuracy.}
\resumeItem{Developed a Gazebo-based virtual testing pipeline to improve line extraction algorithm using hyperparameter tuning.}
\resumeItem{Improved speed of feature-based association algorithm by 50\%, leading to a 30\% faster localization pipeline. Analyzed real-time CPU utilization of SLAM processes and threads using Valgrind, LTTng, and perf.}
\resumeItem{Standardized the pallet pick/drop testing at physical warehouses. The accuracy-repeatability analysis was done using an external laser tracking system by Faro. Also, a procedure to calibrate lidar, steering encoder, and traction encoder was created.}
\resumeItem{Developed a python-based tool to visualize recorded SLAM logs, and automated their offline performance analysis.}
\resumeItem{Certified \href{https://www.youracclaim.com/go/eCNozIcD}{SAFe (Scaled Agile Framework) Practitioner} and trained to use Scrum, Kanban, and XP in a SAFe environment.}
\resumeItemListEnd

\resumeHeading
{Stony Brook University}{Stony Brook, NY}
{Research Assistant}{May 2017 -- Aug 2020}
\resumeItemListStart
\resumeItem{Proposed machine learning and algebraic algorithms for simulation and synthesis of complex single-degree-of-freedom robotic systems, and published multiple articles in journals by the American Society of Mechanical Engineers.}
\resumeItem{Created \href{http://cadcam.eng.sunysb.edu/}{MotionGen}, a web-based mechanism design framework. Uses MEAN (MongoDB, Express.js, Angular.js, Node.js) stack to create a RESTful web service based on MVC architecture. iOS and Android apps created using Apache Cordova framework.}
\resumeItemListEnd
	


%-----------EDUCATION-----------------
\resumeSection{Education}
\resumeHeading
{Stony Brook University}{Stony Brook, NY}
{Ph.D., Mechanical (Concentration: Design and Robotics, Minor: Applied Mathematics), GPA 3.95}{Aug 2015 -- Aug 2020}
\resumeItemListStart
\resumeItem{\textbf{Relevant Courses:} Robotics, Advanced Dynamics, Vibration and Control, Kinematic Analysis and Synthesis,  Applied Stress Analysis, Product Design Optimization, Geometric Modeling, Analysis of Algorithms}
\resumeItemListEnd



%-----------PROJECTS-----------------
\resumeSection{Relevant Projects}
    
    \resumeHeading{Robotics Software Engineer Nanodegree Program}{Udacity}{C\texttt{++}, Python, ROS, Gazebo, AMCL, gmapping, RTABMap}{Apr 2021 -- Apr 2022}
    \resumeItemListStart
    \resumeItem{Simulated Automated Guided vehicles (AGVs) and Autonomous Mobile Robots (AMRs) in a warehouse environment.}
    \resumeItem{Mapped a virtual environment, by manually moving an AGV and creating an occupancy grid map. The AGV only had a 2D lidar sensor on it. Localized an AMR using the same map while controlling it manually or autonomously using path planning. Simulated a complete pick and drop operation. ROS packages like gmapping, AMCL, and move\_base were used.}
    \resumeItem{Used SLAM to manually move an AMR through a virtual environment and create a loop-closed 3D graph map using RTAB-Map. The AMR used a 3D camera and a 2D lidar sensor to create the map for localization.}
    \resumeItemListEnd
    
    \resumeHeading{Self Driving Car Engineer Nanodegree Program}{Udacity}{Python, Jupyter, OpenCV, TensorFlow, Keras, C\texttt{++}, ROS}{Mar 2019 -- Mar 2020}
    \resumeItemListStart
    \resumeItem{Detection: A robust image processing pipeline is created to detect highway lanes in dashcam live-feed.}
    \resumeItem{Perception: Car's position within lane and lane curvature is calculated using bird's eye view (BEV) and polynomial fitting.}
    \resumeItem{Classification: LeNet inspired convolution neural network is developed to detect and classify 40+ kinds of traffic signs.}
    \resumeItem{Deep Learning: Cloned human behavior using an end-to-end neural network to autonomously steer a car using camera input.}
    \resumeItem{Sensor Fusion: Car location is estimated using an extended Kalman filter which acts on LIDAR and RADAR sensor data.}
    \resumeItem{Localization: A 2D particle filter for sparse localization is designed and uses GPS and sensor data with a landmark map.}
    \resumeItem{Trajectory Planning: A Finite State Machine based planner is created to achieve autonomous highway driving with other cars.}
    \resumeItem{Control: A PID controller is implemented to maneuver a vehicle around a virtual track using steering, throttle, and brake.}
    \resumeItem{System Integration: Robot Operation System (ROS) is used to robustly combine Perception, Planning, and Control.}
    \resumeItemListEnd


%-------- SKILLS------------
\resumeSection{Technical Proficiency}
	\vspace{+7pt}
	\resumeItemListStart
	\resumeItem{\textbf{Robotics hardware :}  Nvidia Jetson (Nano and Xavier NX), 2D and 3D Lidar (Sick, Ouster, and Velodyne), RGBD camera (Intel Realsense D455), steering and traction encoder, IMU, Raspberry Pi, Arduino}
	\resumeItem{\textbf{Robotics software :} Keras, Tensorflow, PyTorch, ROS, Gazebo, Rviz, Anaconda, Jupyter, OpenCV, Scikit, Pandas}
	\resumeItem{\textbf{Programming Languages :} C\texttt{++}, Python, Javascript, Matlab, Mathematica, Delphi}
	\resumeItem{\textbf{Tools :} Git, Virtual box, Jenkins (Unit and Integration testing), Msgpack, Valgrind, LTTng}
	\resumeItemListEnd


%-------- RESEARCH PAPERS------------
\resumeSection{Selected Publications}
\vspace{+7pt}
\resumeItemListStart
\resumeItem{Sharma S., Purwar A.; \textbf{A Machine Learning Approach to Solve the Alt–Burmester Problem for Synthesis of Defect-Free Spatial Mechanisms.} ASME J. Computing and Information Science in Engineering; doi:10.1115/1.4051913}
\resumeItem{Sharma S., Purwar A.; \textbf{Path Synthesis of Defect-Free Spatial 5-SS Mechanisms Using Machine Learning.}, ASME IDETC-CIE2020; doi:10.1115/DETC2020-22731}
\resumeItem{Sharma S., Purwar A.; \textbf{Unified Motion Synthesis of Spatial Seven-Bar Platform Mechanisms and Planar-Four Bar Mechanisms.}, ASME IDETC-CIE2020; doi:10.1115/DETC2020-22718}
\resumeItem{Sharma S., Purwar A., Ge Q.J.; \textbf{A Motion Synthesis Approach to Solving Alt-Burmester Problem by Exploiting Fourier Descriptor Relationship Between Path and Orientation.}, ASME J. Mechanisms Robotics; doi:10.1115/1.4042054}
\resumeItem{Sharma S., Purwar A., Ge Q.J.; \textbf{An Optimal Parametrization Scheme for Path Generation Using Fourier Descriptors for Four-Bar Mechanism Synthesis.}, ASME J. Computing and Information Science in Engineering; doi:10.1115/1.4041566}
\resumeItemListEnd

\end{document}
